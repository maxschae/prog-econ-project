\documentclass[11pt, a4paper, leqno]{article}
\usepackage{a4wide}
\usepackage[T1]{fontenc}
\usepackage[utf8]{inputenc}
\usepackage{float, afterpage, rotating, graphicx}
\usepackage{epstopdf}
\usepackage{longtable, booktabs, tabularx}
\usepackage{fancyvrb, moreverb, relsize}
\usepackage{eurosym, calc}
% \usepackage{chngcntr}
\usepackage{amsmath, amssymb, amsfonts, amsthm, bm}
\usepackage{dsfont}
\usepackage{caption}
\usepackage{mdwlist}
\usepackage{xfrac}
\usepackage{setspace}
\usepackage{xcolor}
\usepackage{subcaption}
\usepackage{minibox}
% \usepackage{pdf14} % Enable for Manuscriptcentral -- can't handle pdf 1.5
% \usepackage{endfloat} % Enable to move tables / figures to the end. Useful for some submissions.
\usepackage{algorithm}
\usepackage[noend]{algpseudocode}
\usepackage{float}
\usepackage{ragged2e}

\usepackage{natbib}
\bibliographystyle{rusnat}




\usepackage[unicode=true]{hyperref}
\hypersetup{
    colorlinks=true,
    linkcolor=black,
    anchorcolor=black,
    citecolor=black,
    filecolor=black,
    menucolor=black,
    runcolor=black,
    urlcolor=black
}


\widowpenalty=10000
\clubpenalty=10000

\setlength{\parskip}{1ex}
\setlength{\parindent}{0ex}
\setstretch{1.5}

% number equations within section and subsection
\numberwithin{equation}{section}
\numberwithin{figure}{section}
\numberwithin{table}{section}

% Customise algorithms.
\algnewcommand\Stepone{\item[\textbf{Step 1:}]}
\algnewcommand\Steptwo{\item[\textbf{Step 2:}]}
\algnewcommand\Stepthree{\item[\textbf{Step 3:}]}

\begin{document}

\title{A Comparative Study of Different Estimation Methods in Regression Discontinuity Design\thanks{Caroline Krayer, Max Schäfer, University of Bonn.,  Email: \href{mailto:schaefer.max@gmx.net, caroline.krayer@t-online.de}{\nolinkurl{schaefer [dot] max [at] gmx [dot] net, caroline [dot] krayer [at] t-online [dot] de}}.}}

\author{Caroline Krayer, Max Schäfer}

\date{\today}

\maketitle


\begin{abstract}
	Some abstract here?
\end{abstract}
\thispagestyle{empty}
\addtocounter{page}{-1}
\clearpage

\section{Motivation} % (fold)
\label{sec:motivation}

If you are using this template, please cite this item from the references: \citet{GaudeckerEconProjectTemplates}

In the last years, Regression Discontinuity Designs -- first introduced by \cite{thistlethwaite_campbell} -- have become one of the most popular quasi-experimental methods for causal effect estimation. A large number of studies in economics and social sciences exploit the as good as random assignment of individuals in treatment and control groups to estimate the effect of a (binary) treatment on an outcome of interest. Two common examples are Lee's incumbency study (cf. \cite{lee_2001} and \cite{lee_2007})
and Angrist and Lavy's study of the effect of class sizes on student performance (cf. \cite{angrist_lavy}).

To identify the treatment effect in Regression Discontinuity Designs, non-parametric estimation methods are commonly used as they do not rely on functional form assumptions and thus reduce the risk of bias. But non-parametric methods typically suffer from slower convergence rates and include the necessity of choosing a smoothing parameter -- the bandwidth. In our study, we are interested in the performance difference of non-parametric towards parametric estimation methods. More specifically, we compare local linear regression with two different bandwidth selection procedures and standard global polynomial fitting of varying degrees in different estimation settings by means of a simulation study and real data application.


\section{Theoretical Framework of Regression Discontinuity Design} % (fold)
\label{sec:framework}

Assume that we want to uncover the effect of a binary treatment $ D_{i} \in \lbrace 0,1 \rbrace $ on an outcome of interest $Y_{i}$, where $Y_{i}$(1) and $Y_{i}$(0) denote the potential outcomes if individual $i$ receives treatment or not, respectively, as defined by \cite{rubin}. Let $R_{i}$ be a pre-treatment variable -- the running variable -- which determines the assignment to treatment of each individual. we restrict our attention to the sharp Regression Discontinuity Design where participation in the treatment is mandatory and treatment is granted to those whose value of the running variable passes a fixed cutoff $c$: $D_{i} = \mathds{1}_{R_{i} \geq c}$

Moreover, we assume continuity of the conditional regression functions $\mathbb{E}\left[Y_{i}(0) \vert R_{i} = r\right]$ and  $\mathbb{E}\left[Y_{i}(1) \vert R_{i} = r\right]$ and continuity of the running variable's density $f_{R_{i}}(r)$ at $r=c$. Following \cite{hahn_et_al} we can then identify the average treatment effect as the size of the discontinuity in the conditional expectation of the outcome given the running variable at the cutoff:
\begin{equation}
\tau = \lim_{r \downarrow c} \mathbb{E}\left[Y \vert R = r\right] - \lim_{r \uparrow c} \mathbb{E}\left[ Y \vert R = r\right].
\label{eq: ident_ass}
\end{equation}


% !TEX root = research_paper.tex

\section{Treatment Effect Estimation} % (fold)
\label{sec: estim}

To estimate the causal effect we want to fit a regression function to the data at hand by means of OLS and WLS. Here, we disregard any covariates and regressor values stem from the treatment indicator $D$, running variable $R$ and polynomials thereof. Essentially, we are interested in the difference of the regression function's values when approaching the cutoff from above vis-à-vis below and we allow the regression function to behave differently on the left and right side of the cutoff. Throughout our work, we estimate models of the following kind:

\begin{equation}

Y = \alpha + \tau D + f^{l}(R) + f^{r}(R) + \epsilon .
\label{eq: model_general}

\end{equation}

Under the identification assumptions laid out above $\tau$ recovers the causal effect. Further, $\alpha$ is the intercept and $\epsilon$ constitutes the mean-zero idiosyncratic element. The functions $f^{l}$ and $f^{r}$ are our main objects of interest and we want to investigate how they affect the treatment effect estimator $\tau$.

Identifying causal effects in RDD applications relies by construction on a \textit{locational} feature in the data; treatment assignment discontinuously changes at a specific value of the running variable. This motivates our desire to fit the functional form of the true regression function, especially around the cutoff. In such a setting misspecification of functional forms emerges to be more severe as it may introduce biased estimates of the treatment effect \cite{lee_lemieux_2010}. For intuition suppose the true functional form is nonlinear and we estimate it by means of a linear model. We can still recover a linear prediction which minimizes some criterion, say the sum of squared residuals. While the \textit{best} linear prediction relates to the entire data, at particular points in the regressor's domain we may still be left with serious specification errors. In a RDD study the cutoff may constitute such a specific point, and the estimate would be biased. To reduce specification errors, polynomials of the running variable may be included which can capture (specific) nonlinearities. However, it is not granted that bias due to a misspecified functional form vanishes. Again, note that in RDD the treatment effect inherits a sense of location as it is closely tied to the cutoff determining treatment assignment. Our aim is to get the functional form close to the cutoff right and it is not always clear that data far away from the cutoff can help us to achieve that. In fact, we need to be well informed about the underlying data-generating process to make a case for parametrically estimating the treatment effect using all data available.

Hence, the estimation problem in RDD is widely considered to be addressed by nonparametric methods which relax functional form assumptions and pay closer attention to the treatment effect's locational feature by allowing to restrict data to the cutoff's vicinity, \cite{hahn_et_al}) and \cite{lee_lemieux_2010}. The choice of bandwidth restricts the data to be included in the estimation and is thus crucial as it controls model complexity, that is, a larger bandwidth imposes more structure on the data resulting in higher precision but perhaps bias. Again, this brings up the trade-off between bias and precision which nicely reflects our thinking about global and local models. Moreover, as it is common to believe that data closer to the point of interest depict relationships between the regressor and the response variable which are more accurately describing the relationship at the point of interest compared to points far away, a kernel function is used to assign more weight to observations closer to the point of interest.


\subsection{Parametric Treatment Effect Estimation} % (fold)
\label{sec: param}
In this project we estimate the treatment effect by means of OLS and allow different functional forms on either side of the RDD cutoff. We implement this by interacting the running variable (and polynomials thereof) with the treatment indicator. As common in RDD applications we center the running variable by subtracting the cutoff \cite{lee_lemieux_2010}. The linear model with polynomial degree one thus writes:

\begin{equation}
Y = \alpha + \tau D + \beta_{l} (R-c) + (\beta_{r} - \beta_{l}) (R-c) D + \epsilon .
\label{eq: model_general}
\end{equation}

Similarly, the quadratic model is given by

\begin{equation}
Y = \alpha + \tau D + \beta_{l} (R-c) + (\beta_{r} - \beta_{l}) (R-c) D + \gamma_{l} (R-c)^2 + (\gamma_{r} - \gamma_{l}) (R-c)^2 D + \epsilon .
\end{equation}

In our simulation study and real data application we estimate models with zero up to four polynomial degrees. (Note that a model of zero degree simply returns the difference in means of all individuals below and above the cutoff).



\subsection{Non-parametric Treatment Effect Estimation} % (fold)
\label{sec: non-param}
For estimation by means of local linear regression we can (luckily) rely on general results established by the weighted least squares literature. Here, the weighing scheme is specified by the triangle kernel $K(r) = \max \lbrace 0, 1 - \vert r \vert \rbrace$, a non-negative function with bounded support that assigns a higher weight to observations with regressor values close to the point of interest. \cite{cheng_et_al_1997} show that the triangle kernel is boundary optimal in the sense of minimizing the asymptotic mean squared error at boundary points. As we are interested in estimation at a boundary point -- the RDD cutoff --, we employ the triangle kernel throughout our work. As stated, local linear regressions require a smoothing parameter, the bandwidth, which restricts the data used and we build on two data-driven bandwidth selection procedures. First, the rule-of-thumb bandwidth selection uses the expression for the asymptotically optimal bandwidth and plugs in estimates of unknown quantities \cite{fan_gij_1996}. Second, a bandwidth procedure based on cross-validation which aims to minimize the mean squared error of in-sample predictions. The algorithms required to obtain these bandwidths is discussed in detail below, for implementation details the reader is directed to the project documentation and the available code. Let $h$ be the bandwidth and $\mathds{1}$ depicts the indicator function. We estimate the following pooled regression by means of weighted least squares:


\textit{TODO: how to best present the model in accordance with our implementation...
\begin{equation}
Y = \alpha + \tau D + \beta_{l} (R-c) + (\beta_{r} - \beta_{l}) (R-c) D + \epsilon .

\left( \mathds{1}_{X_{j}>c} \mathds{1}_{r \geq c} + \mathds{1}_{X_{j}<c} \mathds{1}_{r < c} \right) .


\label{eq: model_general}
\end{equation}




\subsection{Parametric or non-parametric?} % (fold)
\label{sec: para discussion}
Parametric and non-parametric treatment effect estimation, as presented here, are fundamentally different since non-parametric methods relax functional form assumptions, constrain the dataset and face different statistical properties. However, in a sense, the non-parametric model can be understood as a generalization of the parametric model since for given data and when choosing the uniform kernel and the bandwidth to coincide with the regressor's support the non-parametric and parametric model are in fact the same. Hence, parametric and non-parametric approaches are sometimes called \textit{global} and \textit{local} methods, respectively. That being said, we want to briefly discuss the trade-offs between these two methods and highlight those in our simulation study and real data application.

The principle trade-off emerges along bias and precision. As stated above, misspecifing the functional form can introduce bias in the treatment estimate in RDD, and we expect flexible non-parametric methods to be better suited to avoid bias. This potential advantage if surely more leveraged in data structures that exhibit stark non-linearities which are hard to capture with polynomial specifications. If we on the other hand are well informed about the true process that generated the data, a treatment effect estimated by a parametric model may not suffer from specification error. In that case, it may even be advisable to use a parametric model as it potentially yields more precise estimates, simply because all data available are used. Also, in settings with very few observations global methods may outperform non-parametric methods as the latter rely on more observations due to their slower convergence rates.








\section{Simulation Study} % (fold)
\label{sec:sim_study}


% !TEX root = research_paper.tex

\section{Simulation Study} % (fold)
\label{sec: sim_study}

We set up a simulation study to assess the performance of global parametric and local non-parametric methods along the bias and precision trade-off for the RDD treatment effect estimate. All estimators are challenged for three different data-generating processes (DGPs). The first DGP is based on a linear model, the second wave of datasets stem from a polynomial process of order four, and finally we relate the running variable and outcomes with help of sinus, co-sinus and polynomial functions. Randomness is introduced by an idiosyncratic element drawn from a mean-zero normal distribution with fixed variance. The data-generating processes are illustrated by FIGURE where a very small random noise component is specified and observations are collapsed into bins containing averages for reasons of illustration.

Our results are based on 250 randomly drawn datasets with 500 observations each per DGP, and in total we draw 750 datasets. Aside from the average estimated treatment effects, we compute the estimates' standard deviation, coverage probability and mean squared error. The coverage probability informs us about the share of estimate's 95-percent confidence bands which cover the true treatment effect. Repeating the simulation infinitely many times, the coverage probability of an unbiased estimator approaches 95 percent if the significance level is chosen to be five percent. While we can infer information on the bias the coverage probability does not inform us about the estimator's precision -- i.e. an imprecisely measured coefficient has larger confidence intervals which will likely (here, with an observed share of 95 percent if it is estimated without bias) cover the true coefficient. Parametric treatment effect estimation is conducted for models with polynomial degrees varying from zero\footnote{For a model of polynomial degree of zero the treatment effect estimate is a simple difference in means of outcomes left vis-à-vis right of the cutoff.} to five. For local linear regression we report results for different bandwidths based on leave-one-out cross-validation, the rule-of-thumb selection procedure and 50 percent under- as well as 200 percent oversmoothing thereof. Note that 50 and 200 percent of the rule-of-thumb bandwidth correspond to the grid's lower and upper bound where the cross-validation bandwidth is chosen from.

TABLE X shows results for parametrically estimating the treatment effect for data coming from the three DGPs. As expected the linear global model is well suited to recover the treatment effect of .75, and so are higher order polynomials -- this is foremost the case since higher order polynomial models comprise the (true) linear model as a special case. The respective coverage probabilities hover around the expected value of 95 and we would need to increase the Monte Carlo simulations to approach .95 precisely. In line with econometric theory the standard deviation and mean squared error as measures on precision increase in value with every extra polynomial specification [SOURCE?]. Simply comparing means unsurprisingly overestimates the treatment effect since we have a larger slope coefficient on the cutoff's right side. Panel (b) of TABLE X exhibits results for the \textit{polynomial} DGP and we see that starting with the cubic model higher order polynomials estimate the treatment effect consistently. As was noted before, increasing the number of regressors through additional polynomials makes the treatment effect estimate less precise indicated by larger standard deviations and mean squared errors. The linear and quadratic specification perform rather poorly exhibiting large bias and variance. The drawbacks of inflexible functional form assumptions and the restriction to regard all data globally available become apparent by glancing at results in Panel C. While the linear and cubic model are already off by quite a bit, other parametric specifications are not able to resemble their estimates around the true value. Without prior knowledge of the true data-generating process the researcher may collect visual signs of the observed data but can hardly be guided by the present results.

Similarly, we collect performance measures for estimates from local linear regression for four bandwidths in TABLE Y. When consulting the results, three remarks claim attention. First, restricting the data through the bandwidth pays off in terms of bias. The rule-of-thumb, its undersmoothed companion and the cross-validated bandwidth all lead to consistent results in terms of estimated average treatment effects, with its strongest game for the linear model -- perhaps not to the reader's surprise as we locally fit a linear model. Only the oversmoothed rule-of-thumb bandwidth struggles to assist the estimator to recover on average the true treatment effect, in the polynomial and non-parametric DGPs. Second, the coverage probabilities across models and DGPs lacks behind the targeted 95 percent. Third, a larger bandwidth increases precision and tends to introduce bias. To make this claim we compare the undersmoothed against the oversmoothed rule-of-thumb bandwidth and notice that both precision measures are smaller choosing a larger bandwidth. This comes at the cost of introducing bias in the estimate for the second and third DGP. For the first DGP the bias is not present since the underlying true model is linear throughout, and we notice that the estimates with the larger bandwidth are on average more precise. In fact, when comparing the results from the local linear and the global linear model (for the linear DGP) we find that the global linear model is most precise -- it features the maximum bandwidth by construction.

In all, we find in our simulation study that parametric methods are consistently estimating the treatment effect in scenarios where the true data-generating process is based on a polynomial relationship between the running variable and the outcome. They perform poorly for data with less structure, if no classic polynomials are involved or the regression model features too few polynomial degrees to cope with richer data structures. While non-parametric methods are capturing linear and polynomial relationships in data they really shine -- in comparison to global methods -- if the underlying data-generating processes become more involved. Choosing a smaller bandwidth improves consistency at the cost of precision. The key take-away from this investigation is: local methods face less risk of introducing bias to estimates and can handle more complex data structures. However, if the underlying data-generating process is well-known tailored global methods may be used to increase precision.



TODO: Table 5.3??











\begin{figure}[H]
	\centering
	\includegraphics[width=\textwidth]{../../out/figures/simulation_study/simulated_rdd_graphs.png}
	\caption{\textsc{Visualisation of Data-Generating Process}}
	\label{fig: dgp}
\end{figure}

\begin{table}[H]
	\begin{subtable}{\textwidth}
		\centering
		\input{../../out/tables/simulation_study/perf_meas_table_linear_p.tex}
		\caption{Linear DGP}
		\label{tab: global_poly_linear}
		\hspace{\fill}
	\end{subtable}
	\begin{subtable}{\textwidth}
		\centering
		\input{../../out/tables/simulation_study/perf_meas_table_poly_p.tex}
		\caption{Polynomial DGP}
		\label{tab: global_poly_poly}
		\hspace{\fill}
	\end{subtable}
	\begin{subtable}{\textwidth}
		\centering
		\input{../../out/tables/simulation_study/perf_meas_table_nonparametric_p.tex}
		\caption{Non-parametric DGP}
		\label{tab: global_poly_nonparam}
	\end{subtable}
	\caption{\textsc{Performance of Global Polynomial Estimators}}
	\label{tab: perf_para}
\end{table}


\begin{table}[H]
	\begin{subtable}{\textwidth}
		\centering
		\input{../../out/tables/simulation_study/perf_meas_table_linear_np.tex}
		\caption{Linear DGP}
		\label{tab: llr_linear}
		\hspace{\fill}
	\end{subtable}
	\begin{subtable}{\textwidth}
		\centering
		\input{../../out/tables/simulation_study/perf_meas_table_poly_np.tex}
		\caption{Polynomial DGP}
		\label{tab: llr_poly}
		\hspace{\fill}
	\end{subtable}
	\begin{subtable}{\textwidth}
		\centering
		\input{../../out/tables/simulation_study/perf_meas_table_nonparametric_np.tex}
		\caption{Non-Parametric DGP}
		\label{tab: llr_nonparam}
	\end{subtable}
	\caption{\textsc{Performance of Local Linear Regression}}
	\label{tab: perf_nonpara}
\end{table}


\begin{table}[H]
	\begin{subtable}{\textwidth}
		\centering
		\input{../../out/tables/simulation_study/bw_select_table_linear_np.tex}
		\caption{Linear DGP}
		\label{tab: bw_perf_linear}
		\hspace{\fill}
	\end{subtable}
	\begin{subtable}{\textwidth}
		\centering
		\input{../../out/tables/simulation_study/bw_select_table_poly_np.tex}
		\caption{Polynomial DGP}
		\label{tab: bw_perf_poly}
		\hspace{\fill}
	\end{subtable}
	\begin{subtable}{\textwidth}
		\centering
		\input{../../out/tables/simulation_study/bw_select_table_nonparametric_np.tex}
		\caption{Non-Parametric DGP}
		\label{tab: bw_perf_nonparam}
	\end{subtable}
	\caption{\textsc{Performance of Bandwidth Selection Procedures}}
\end{table}




\begin{table}[H]
	\begin{subtable}{\textwidth}
		\centering
		\input{../../out/tables/simulation_study/perf_meas_table_linear_np_discr_500.tex}
		\caption{500 Observations}
		\label{tab: llr_500}
		\hspace{\fill}
	\end{subtable}
	\begin{subtable}{\textwidth}
		\centering
		\input{../../out/tables/simulation_study/perf_meas_table_linear_np_discr_200.tex}
		\caption{200 Observations}
		\label{tab: llr_200}
		\hspace{\fill}
	\end{subtable}
	\caption{\textsc{Performance of Local Linear Regression on Discrete Data}}
	\label{tab: perf_nonpara_discr}
\end{table}


\begin{table}[H]
	\begin{subtable}{\textwidth}
		\centering
		\input{../../out/tables/simulation_study/perf_meas_table_linear_p_discr_500.tex}
		\caption{500 Observations}
		\label{tab: global_poly_500}
		\hspace{\fill}
	\end{subtable}
	\begin{subtable}{\textwidth}
		\centering
		\input{../../out/tables/simulation_study/perf_meas_table_linear_p_discr_200.tex}
		\caption{200 Observations}
		\label{tab: global_poly_200}
		\hspace{\fill}
	\end{subtable}
	\caption{\textsc{Performance of Global Polynomial Estimators on Discrete Data}}
	\label{tab: perf_para_discr}
\end{table}





% !TEX root = research_paper.tex

\section{Data Application} % (fold)
\label{sec: data_application}

To compare the performance of the above considered parametric and non-parametric estimation methods in large samples and real world applications, we use the labour market study of \cite{nekoei_weber} that exploits an aged-based Regression Discontinuity Design to estimate the effects of extended unemployment benefits on non-employment duration and re-employment wages, respectively.

The study exploits a special feature of the Austrian unemployment insurance system. Normally, workers who have been employed for 3 years during the last 5 years up to the layoff date are eligible for 30 weeks of unemployment benefits. But since August, 1, 1989, once their age passes the cutoff of 40 years at the time of layoff, the workers are eligible for a benefit extension to 39 weeks, provided they have worked for 6 years during the last 10 years. The cutoff at the age of 40 thus creates a natural distinction between treatment and control groups.

For their empirical analysis, the authors combine two different datasets. The Austrian Social Security Database provides daily employment records and annual earnings by employer for all private sector employees. These are matched with Austrian Unemployment Registers at the individual level that include data on unemployment spells and benefit receipts. Furthermore, the sample is restricted to workers aged 30 to 50 years who have been laid off after the introduction of the law on August, 1, 1989 and are eligible for extended unemployment benefits provided their age passes the cutoff. So all in all, the data set used for the analysis comprises 1,738,787 individual observations between 1989 and 2011.

The variables that we need in our analysis are measured as follows. The running variable \textit{age} is defined as the exact age at the date of layoff up to five decimal places. \textit{Non-employment duration} is measured as the number of days between the end of a lost job and the start of a new job. The \textit{daily wage rate} is defined as the worker's annual earnings per employer divided by the number of days he has worked for him. It is then used to compute the \textit{wage change} between pre- and post-unemployment jobs as the log difference between the daily wage in the year of separation and in the year when the new job starts.

In order to get a first intuition of the setting, Figure \ref{fig: estim_ui_benefits} plots the running variable age, grouped in quarterly age groups, against the average of the variables non-employment duration and wage change in these groups, respectively. We further add the fitted lines of quadratic polynomials to get a better sense of a potential discontinuity gap. Indeed, both panels give a first evidence of a discontinuity gap in the conditional expectation of the outcomes given the running variable at the cutoff of 40 years. Both the effect of extended unemployment benefits on non-employment duration and wage change seem to be positive.

\begin{figure}[H]
	\centering
	\includegraphics[width=\textwidth]{../../out/figures/data_analysis/rdd_graphs.png}
	\caption{\textsc{Non-Employment Duration and Wage Change as a Function of Age}}
	\label{fig: estim_ui_benefits}
	\medskip
	\justify
	\footnotesize{Notes: The variable age is grouped in bins covering 4 months and the average value in these groups is plotted against the average value of the outcome variable. A quadratic fit is added to the data. The outcome variable in Panel A is non-employment duration, measured as the number of days between two consecutive jobs. The outcome variable in Panel B is wage change, computed as the log difference between the daily wage in the year of separation and in the year when the new job starts.}
\end{figure}

To assess the effect in more detail and to compare the performance of our studied methods, Table \ref{tab: estim_ui_benefits} contains estimation results for both outcome variables using global polynomials of degree 0 to 4 and local linear regressions with the bandwidth selection procedures described in Section \ref{sec: estim}. The grid of bandwidths used as an input parameter in the cross-validation algorithm takes the rule-of-thumb bandwidth as a reference and contains 32 equally spaced values ranging from a minimum of half of the rule-of-thumb-bandwidth up to a maximum value of 10. Thereby, a value of 10 corresponds to taking the whole range of data into account. As the data set contains more than 1.5 million observations, the computationally expensive leave-one-out cross-validation cannot be performed with all observations as described in Algorithm \ref{alg:cv}. As a solution, we randomly sample 2000 observations within a range of 3 years near the cutoff on which the cross-validation algorithm is then applied. The results in Table \ref{tab: estim_ui_benefits} reveal that -- except of a simple comparison in means (polynomial of order 0) -- all methods estimate similar treatment effects for both variables of interest.


\begin{table}[H]
	\centering
	\hspace{\fill}
	\input{../../out/tables/data_analysis/reproduce_main_results_table_2.tex}
	\caption{\textsc{Estimated Effect of Unemployment Benefit Extension}}
	\label{tab: estim_ui_benefits}
	\medskip
	\justify
	\footnotesize{Notes: The treatment effect is estimated parametrically using global polynomial fitting with varying coefficients on either side of the cutoff and polynomial degrees of order 0 to 4 as well as non-parametrically using local linear regression with bandwidths selected with leave-one-out cross-validation or the rule-of-thumb procedure. The standard errors stem from pooled regressions.}
\end{table}

The effect of extended unemployment benefits on non-employment duration is estimated at an average of one to two days, indicating that people receiving unemployment benefits for a longer time period are tending to stay unemployed for a longer spell as well. The results for polynomials of degree 0 to 2 and the non-parametric methods are significant at the 1-percent level, whereas the results for polynomials of order 3 and 4 are only significant at the 5- and 10-percent level, respectively. As already revealed in our simulation study, the standard errors for global polynomial fitting increase with the degree of polynomial used, explaining the larger p-values. Comparing the non-parametric methods, the two bandwidth selection procedures used lead to slightly different estimation results, as the treatment effect estimated with local linear regression and the bandwidth selected by cross-validation leads to a smaller estimate and standard error.

The effect of extended unemployment benefits on the log difference between daily wages in the old and new job is estimated positively as well at an average of 0.004 for global polynomials of degree 1 to 3 and local linear regression using the bandwidth selected by cross-validation. This means, that the re-employment wage is estimated as on average 0.4$\%$ higher for people receiving extended unemployment benefits. The p-values are far larger such that only the polynomial degrees of order 0 to 2 return estimates significant at the 1-percent level. A treatment effect estimation in terms of comparison in means is far off again and even estimates the effect to be negative. Interestingly, when comparing the non-parametric estimation methods, the rule-of-thumb bandwidth selection procedure performs quite differently. It estimates the effect to be nearly zero and is clearly not significant at any conventional level.

As the treatment effect estimates for both outcome variables differ when using the rule-of-thumb bandwidth selection procedure compared to cross-validation in local linear regression, Figure \ref{fig: bw_perf} finally compares the numeric values of the bandwidths selected and plots the treatment effect estimates as well as the 95-percent confidence intervals as a function of the bandwidth. We see that cross-validation always selects the largest possible bandwidth of 10 whereas the rule-of-thumb procedure leads to significantly smaller bandwidths -- 4.64 when estimating the effect on non-employment duration and 1.01 when estimating the effect on re-employment wage. For the outcome non-employment duration, the treatment effect estimated non-parametrically with any of the considered bandwidths always ranges between the one estimated with a linear model and the one estimated with a quadratic polynomial. It slightly increases as a function of the bandwidth until a value of about 6 and then starts marginally decreasing again. When estimating the effect on wage change however, there is a sharp decrease followed by a sharp increase in the treatment effect estimate for small bandwidths and afterwards, the estimate stabilises at the value predicted by the linear and quadratic model as well. In this setting, the rule-of-thumb bandwidth is close to the value minimising the treatment effect estimate.

\begin{figure}[H]
	\centering
	\includegraphics[width=\textwidth]{../../out/figures/data_analysis/treatment_effect_estimates.png}
	\caption{\textsc{Performance of Local Linear Regression with Different Bandwidths.}}
	\label{fig: bw_perf}
	\medskip
	\justify
	\footnotesize{Notes: The bandwidth used in local linear regression is plotted against the corresponding treatment effect estimate and 95-percent confidence intervals. The outcome variable in Panel A is non-employment duration, measured as the number of days between two consecutive jobs. The outcome variable in Panel B is wage change, computed as the log difference between the daily wage in the year of separation and in the year when the new job starts.}
\end{figure}

These findings again highlight that depending on the underlying data generating process, the estimated treatment effect can depend largely on the method used and among non-parametric methods the bandwidth may have a significant impact. Therefore, it is very important to be not too restrictive when choosing the model used for estimation and possibly even report results or various procedures that are in accordance with the visual appearance of the data at hand.





% !TEX root = research_paper.tex

\section{Conclusion} % (fold)
\label{sec: conclusion}



The principle trade-off emerges along bias and precision. As stated above, misspecifying the functional form can introduce bias in the treatment estimate in RDD, and we expect flexible non-parametric methods to be better suited to avoid bias. This potential advantage if surely more leveraged in data structures that exhibit stark non-linearities which are hard to capture with polynomial specifications. If we on the other hand are well informed about the true process that generated the data, a treatment effect estimated by a parametric model may not suffer from specification error. In that case, it may even be advisable to use a parametric model as it potentially yields more precise estimates, simply because all data available are used. Also, in settings with very few observations global methods may outperform non-parametric methods as the latter rely on more observations due to their slower convergence rates.




% Add references.
\clearpage
\bibliography{refs}




% \appendix

% The chngctr package is needed for the following lines.
% \counterwithin{table}{section}
% \counterwithin{figure}{section}

\end{document}
