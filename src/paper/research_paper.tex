\documentclass[11pt, a4paper, leqno]{article}
\usepackage{a4wide}
\usepackage[T1]{fontenc}
\usepackage[utf8]{inputenc}
\usepackage{float, afterpage, rotating, graphicx}
\usepackage{epstopdf}
\usepackage{longtable, booktabs, tabularx}
\usepackage{fancyvrb, moreverb, relsize}
\usepackage{eurosym, calc}
% \usepackage{chngcntr}
\usepackage{amsmath, amssymb, amsfonts, amsthm, bm}
\usepackage{caption}
\usepackage{mdwlist}
\usepackage{xfrac}
\usepackage{setspace}
\usepackage{xcolor}
\usepackage{subcaption}
\usepackage{minibox}
% \usepackage{pdf14} % Enable for Manuscriptcentral -- can't handle pdf 1.5
% \usepackage{endfloat} % Enable to move tables / figures to the end. Useful for some submissions.



\usepackage{natbib}
\bibliographystyle{rusnat}




\usepackage[unicode=true]{hyperref}
\hypersetup{
    colorlinks=true,
    linkcolor=black,
    anchorcolor=black,
    citecolor=black,
    filecolor=black,
    menucolor=black,
    runcolor=black,
    urlcolor=black
}


\widowpenalty=10000
\clubpenalty=10000

\setlength{\parskip}{1ex}
\setlength{\parindent}{0ex}
\setstretch{1.5}


\begin{document}

\title{Different Estimation Methods in Regression Discontinuity Design\thanks{Caroline Krayer, Max Schäfer, University of Bonn. Email: \href{mailto:schaefer.max@gmx.net}{\nolinkurl{schaefer [dot] max [at] gmx [dot] net}}.}}

\author{Caroline Krayer, Max Schäfer}

\date{
{\bf Preliminary -- please do not quote}
\\[1ex]
\today
}

\maketitle


\begin{abstract}
	Some abstract here.
\end{abstract}
\clearpage

\section{Introduction} % (fold)
\label{sec:introduction}

If you are using this template, please cite this item from the references: \citet{GaudeckerEconProjectTemplates}

% section introduction (end)



\section{Theoretical Framework and Identification} % (fold)
\label{sec:framework}

Can use quite a bit from our research module paper.



\section{Simulation Study} % (fold)
\label{sec:sim_study}



\begin{table}
	\centering

	\input{perform_meas_table_linear_parametric_discrete_False_c_1}

	\caption{\textsc{Performance Measures of OLS Regression}}
	\label{tab:perform_meas_table_linear_parametric_discrete_False_c_1}
	\medskip
	\raggedright
	\footnotesize
	\textit{Notes:} Treatment effect estimate and its performance measure for the linear potential outcomes model based on continuous data. Estimation follows general OLS regression with varying polynomial degrees. True treatment effect is ten and the cutoff is chosen such that few observations are adjacent to it. \\
\end{table}


\begin{table}
	\centering

	\input{perform_meas_table_linear_parametric_discrete_False_c_10}

	\caption{\textsc{Performance Measures of OLS Regression}}
	\label{tab:perform_meas_table_linear_parametric_discrete_False_c_10}
	\medskip
	\raggedright
	\footnotesize
	\textit{Notes:} Treatment effect estimate and its performance measure for the linear potential outcomes model based on continuous data. Estimation follows general OLS regression with varying polynomial degrees. True treatment effect is ten and the cutoff is chosen such that many observations are adjacent to it. \\
\end{table}


\begin{table}
	\centering

	\input{perform_meas_table_linear_parametric_discrete_False}

	\caption{\textsc{Performance Measures of OLS Regression}}
	\label{tab:perform_meas_table_linear_parametric_discrete_False}
	\medskip
	\raggedright
	\footnotesize
	\textit{Notes:} Treatment effect estimate and its performance measure for the linear potential outcomes model based on continuous data. Estimation follows general OLS regression with varying polynomial degrees. True treatment effect is ten. \\
\end{table}


\begin{table}
	\centering

	\input{perform_meas_table_linear_parametric_discrete_True}

	\caption{\textsc{Performance Measures of OLS Regression}}
	\label{tab:perform_meas_table_linear_parametric_discrete_True}
	\medskip
	\raggedright
	\footnotesize
	\textit{Notes:} Treatment effect estimate and its performance measure for the linear potential outcomes model based on discrete data. Estimation follows general OLS regression with varying polynomial degrees. True treatment effect is ten. \\
\end{table}


\begin{table}
	\centering

	\input{perform_meas_table_linear_nonparametric_discrete_True}

	\caption{\textsc{Performance Measures of Local Linear Regression}}
	\label{tab:perform_meas_table_linear_nonparametric_discrete_True}
	\medskip
	\raggedright
	\footnotesize
	\textit{Notes:} Local linear treatment effect estimate and its performance measure for the linear potential outcomes model based on discrete data. True treatment effect is ten. \\
\end{table}



\begin{table}
\centering

	\input{perform_meas_table_linear_nonparametric_discrete_False}

\caption{\textsc{Performance Measures of Local Linear Regression}}
\label{tab:perform_meas_table_linear_nonparametric_discrete_False}
\medskip
\raggedright
\footnotesize
\textit{Notes:} Local linear treatment effect estimate and its performance measure for the linear potential outcomes model based on continuous data. True treatment effect is ten. \\
\end{table}


\begin{table}
	\centering

	\input{perform_meas_table_linear_nonparametric_discrete_False_c_1}

	\caption{\textsc{Performance Measures of Local Linear Regression}}
	\label{tab:perform_meas_table_linear_nonparametric_discrete_False_c_1}
	\medskip
	\raggedright
	\footnotesize
	\textit{Notes:} Local linear treatment effect estimate and its performance measure for the linear potential outcomes model based on continuous data. True treatment effect is ten and the cutoff is chosen such that few observations are adjacent to it. \\
\end{table}


\begin{table}
	\centering

	\input{perform_meas_table_linear_nonparametric_discrete_False_c_10}

	\caption{\textsc{Performance Measures of Local Linear Regression}}
	\label{tab:perform_meas_table_linear_nonparametric_discrete_False_c_10}
	\medskip
	\raggedright
	\footnotesize
	\textit{Notes:} Local linear treatment effect estimate and its performance measure for the linear potential outcomes model based on continuous data. True treatment effect is ten and the cutoff is chosen such that many observations are adjacent to it. \\
\end{table}


\bibliography{refs}



% \appendix

% The chngctr package is needed for the following lines.
% \counterwithin{table}{section}
% \counterwithin{figure}{section}

\end{document}
