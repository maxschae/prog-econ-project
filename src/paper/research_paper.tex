\documentclass[11pt, a4paper, leqno]{article}
\usepackage{a4wide}
\usepackage[T1]{fontenc}
\usepackage[utf8]{inputenc}
\usepackage{float, afterpage, rotating, graphicx}
\usepackage{epstopdf}
\usepackage{longtable, booktabs, tabularx}
\usepackage{fancyvrb, moreverb, relsize}
\usepackage{eurosym, calc}
% \usepackage{chngcntr}
\usepackage{amsmath, amssymb, amsfonts, amsthm, bm}
\usepackage{dsfont}
\usepackage{caption}
\usepackage{mdwlist}
\usepackage{xfrac}
\usepackage{setspace}
\usepackage{xcolor}
\usepackage{subcaption}
\usepackage{minibox}
% \usepackage{pdf14} % Enable for Manuscriptcentral -- can't handle pdf 1.5
% \usepackage{endfloat} % Enable to move tables / figures to the end. Useful for some submissions.


\usepackage{natbib}
\bibliographystyle{rusnat}




\usepackage[unicode=true]{hyperref}
\hypersetup{
    colorlinks=true,
    linkcolor=black,
    anchorcolor=black,
    citecolor=black,
    filecolor=black,
    menucolor=black,
    runcolor=black,
    urlcolor=black
}


\widowpenalty=10000
\clubpenalty=10000

\setlength{\parskip}{1ex}
\setlength{\parindent}{0ex}
\setstretch{1.5}


\begin{document}

\title{Different Estimation Methods in Regression Discontinuity Design\thanks{Caroline Krayer, Max Schäfer, University of Bonn. Email: \href{mailto:schaefer.max@gmx.net}{\nolinkurl{schaefer [dot] max [at] gmx [dot] net}}.}}

\author{Caroline Krayer, Max Schäfer}

\date{
{\bf Preliminary -- please do not quote}
\\[1ex]
\today
}

\maketitle


\begin{abstract}
	Some abstract here?
\end{abstract}
\clearpage

\section{Motivation} % (fold)
\label{sec:motivation}

If you are using this template, please cite this item from the references: \citet{GaudeckerEconProjectTemplates}

In the last years, Regression Discontinuity Designs -- first introduced by Thistlethwaite and Campbell in 1960 (\cite{thistlethwaite_campbell}) -- have become one of the most popular quasi-experimental methods for causal effect estimation. A large number of studies in economics and social sciences exploit the as good as random assignment of individuals in treatment and control groups to estimate the effect of a (binary) treatment on an outcome of interest. Two common examples are Lee's incumbency study (cf. \cite{lee_2001} and \cite{lee_2007})
and Angrist and Lavy's study of the effect of class sizes on student performance (cf. \cite{angrist_lavy}).

To identify the treatment effect in Regression Discontinuity Designs, non-parametric estimation methods are commonly used as they do not rely on functional form assumptions and thus reduce the risk of bias. But non-parametric methods typically suffer from slower convergence rates and include the necessity of choosing a smoothing parameter -- the bandwidth. In our study, we are interested in the performance difference of non-parametric towards parametric estimation methods. More specifically, we compare local linear regression with two different bandwidth selection procedures and standard global polynomial fitting of varying degrees in different estimation settings by means of a simulation study and real data application.


\section{Theoretical Framework} % (fold)
\label{sec:framework}

Assume that we want to uncover the effect of a binary treatment $ D_{i} \in \lbrace 0,1 \rbrace $ on an outcome of interest $Y_{i}$, where $Y_{i}$(1) and $Y_{i}$(0) denote the potential outcomes if individual $i$ receives treatment or not, respectively, as defined by Rubin \cite{rubin}. Let $R_{i}$ be a pre-treatment variable -- the running variable -- which determines the assignment to treatment of each individual. we restrict our attention to the sharp Regression Discontinuity Design where participation in the treatment is mandatory and treatment is granted to those whose value of the running variable passes a fixed cutoff $c$: $D_{i} = \mathds{1}_{R_{i} \geq c}$

Moreover, we assume continuity of the conditional regression functions $\mathbb{E}\left[Y_{i}(0) \vert R_{i} = r\right]$, $\mathbb{E}\left[Y_{i}(1) \vert R_{i} = r\right]$ and the density of the running variable $f_{R_{i}}(r)$ at $r=c$. Following \cite{hahn_et_al} we can then identify the average treatment effect as the size of the discontinuity in the conditional expectation of the outcome given the running variable:
\begin{equation*}
\tau = \lim_{r \downarrow c} \mathbb{E}\left[Y \vert R = r\right] - \lim_{r \uparrow c} \mathbb{E}\left[ Y \vert R = r\right].
\end{equation*}


\section{Parametric Treatment Effect Estimation} % (fold)
\label{sec: param}



\section{Non-parametric Treatment Effect Estimation} % (fold)
\label{sec: non-param}


\section{Simulation Study} % (fold)
\label{sec:sim_study}






\bibliography{refs}




% \appendix

% The chngctr package is needed for the following lines.
% \counterwithin{table}{section}
% \counterwithin{figure}{section}

\end{document}
