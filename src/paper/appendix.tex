% !TEX root = research_paper.tex


\usepackage[toc, page]{appendix}

\begin{appendices}
\counterwithin{figure}{subsection}
\counterwithin{table}{subsection}
\subsection{Simulation Study: Results for Discrete Data} \label{discrete_data}

\begin{table}[H]
	\begin{subtable}{\textwidth}
		\centering
		\input{../../out/tables/simulation_study/perf_meas_table_linear_np_discr_500.tex}
		\caption{500 Observations}
		\label{tab: llr_500}
		\hspace{\fill}
	\end{subtable}
	\begin{subtable}{\textwidth}
		\centering
		\input{../../out/tables/simulation_study/perf_meas_table_linear_np_discr_200.tex}
		\caption{200 Observations}
		\label{tab: llr_200}
		\hspace{\fill}
	\end{subtable}
	\caption{\textsc{Performance of Local Linear Regression on Discrete Data}}
	\label{tab: perf_nonpara_discr}
\end{table}


\begin{table}[H]
	\begin{subtable}{\textwidth}
		\centering
		\input{../../out/tables/simulation_study/perf_meas_table_linear_p_discr_500.tex}
		\caption{500 Observations}
		\label{tab: global_poly_500}
		\hspace{\fill}
	\end{subtable}
	\begin{subtable}{\textwidth}
		\centering
		\input{../../out/tables/simulation_study/perf_meas_table_linear_p_discr_200.tex}
		\caption{200 Observations}
		\label{tab: global_poly_200}
		\hspace{\fill}
	\end{subtable}
	\caption{\textsc{Performance of Global Polynomial Estimators on Discrete Data}}
	\label{tab: perf_para_discr}
\end{table}




\end{appendices}