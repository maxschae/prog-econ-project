% !TEX root = research_paper.tex

\section{Conclusion} % (fold)
\label{sec: conclusion}



The principle trade-off emerges along bias and precision. As stated above, misspecifying the functional form can introduce bias in the treatment estimate in RDD, and we expect flexible non-parametric methods to be better suited to avoid bias. This potential advantage if surely more leveraged in data structures that exhibit stark non-linearities which are hard to capture with polynomial specifications. If we on the other hand are well informed about the true process that generated the data, a treatment effect estimated by a parametric model may not suffer from specification error. In that case, it may even be advisable to use a parametric model as it potentially yields more precise estimates, simply because all data available are used. Also, in settings with very few observations global methods may outperform non-parametric methods as the latter rely on more observations due to their slower convergence rates.
