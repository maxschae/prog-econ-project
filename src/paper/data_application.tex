% !TEX root = research_paper.tex

\section{Data Application} % (fold)
\label{sec: data_application}

To compare the performance of the above considered parametric and non-parametric estimation methods in large samples and real world applications, we use the labour market study of \cite{nekoei_weber} that exploits an aged-based Regression Discontinuity Design to estimate the effects of extended unemployment benefits on non-employment duration and re-employment wages, respectively.

The study exploits a special feature of the Austrian unemployment insurance system. Normally, workers who have been employed for 3 years during the last 5 years up to the layoff date are eligible for 30 weeks of unemployment benefits. But since August, 1, 1989, once their age passes the cutoff of 40 years at the time of layoff, the workers are eligible for a benefit extension to 39 weeks, provided they have worked for 6 years during the last 10 years. The cutoff at the age of 40 thus creates a natural distinction between treatment and control groups.

For their empirical analysis, the authors combine two different datasets. The Austrian Social Security Database provides daily employment records and annual earnings by employer for all private sector employees. These are matched with Austrian Unemployment Registers at the individual level that include data on unemployment spells and benefit receipts. Furthermore, the sample is restricted to workers aged 30 to 50 years who have been laid off after the introduction of the law on August, 1, 1989 and are eligible for extended unemployment benefits provided their age passes the cutoff. So all in all, the data set used for the analysis comprises 1,738,787 individual observations between 1989 and 2011.

The variables that we need in our analysis are measured as follows. The running variable \textit{age} is defined as the exact age at the date of layoff up to five decimal places. \textit{Non-employment duration} is measured as the number of days between the end of a lost job and the start of a new job. The \textit{daily wage rate} is defined as the worker's annual earnings per employer divided by the number of days he has worked for him. It is then used to compute the \textit{wage change} between pre- and post-unemployment jobs as the log difference between the daily wage in the year of separation and in the year when the new job starts.

In order to get a first intuition of the setting, Figure \ref{fig: estim_ui_benefits} plots the running variable age, grouped in quarterly age groups, against the average of the variables non-employment duration and wage change in these groups, respectively. We further add the fitted lines of quadratic polynomials to get a better sense of a potential discontinuity gap. Indeed, both panels give a first evidence of a discontinuity gap in the conditional expectation of the outcomes given the running variable at the cutoff of 40 years. Both the effect of extended unemployment benefits on non-employment duration and wage change seem to be positive.

\begin{figure}[H]
	\centering
	\includegraphics[width=\textwidth]{../../out/figures/data_analysis/rdd_graphs.png}
	\caption{\textsc{Non-Employment Duration and Wage Change as a Function of Age}}
	\label{fig: estim_ui_benefits}
	\medskip
	\justify
	\footnotesize{Notes: The variable age is grouped in bins covering 4 months and the average value in these groups is plotted against the average value of the outcome variable. A quadratic fit is added to the data. The outcome variable in Panel A is non-employment duration, measured as the number of days between two consecutive jobs. The outcome variable in Panel B is wage change, computed as the log difference between the daily wage in the year of separation and in the year when the new job starts.}
\end{figure}

To assess the effect in more detail and to compare the performance of our studied methods, Table \ref{tab: estim_ui_benefits} contains estimation results for both outcome variables using global polynomials of degree 0 to 4 and local linear regressions with the bandwidth selection procedures described in Section \ref{sec: estim}. The grid of bandwidths used as an input parameter in the cross-validation algorithm takes the rule-of-thumb bandwidth as a reference and contains 32 equally spaced values ranging from a minimum of half of the rule-of-thumb-bandwidth up to a maximum value of 10. Thereby, a value of 10 corresponds to taking the whole range of data into account. As the data set contains more than 1.5 million observations, the computationally expensive leave-one-out cross-validation cannot be performed with all observations as described in Algorithm \ref{alg:cv}. As a solution, we randomly sample 2000 observations within a range of 3 years near the cutoff on which the cross-validation algorithm is then applied. The results in Table \ref{tab: estim_ui_benefits} reveal that -- except of a simple comparison in means (polynomial of order 0) -- all methods estimate similar treatment effects for both variables of interest.


\begin{table}[H]
	\centering
	\hspace{\fill}
	\input{../../out/tables/data_analysis/reproduce_main_results_table_2.tex}
	\caption{\textsc{Estimated Effect of Unemployment Benefit Extension}}
	\label{tab: estim_ui_benefits}
	\medskip
	\justify
	\footnotesize{Notes: The treatment effect is estimated parametrically using global polynomial fitting with varying coefficients on either side of the cutoff and polynomial degrees of order 0 to 4 as well as non-parametrically using local linear regression with bandwidths selected with leave-one-out cross-validation or the rule-of-thumb procedure. The standard errors stem from pooled regressions.}
\end{table}

The effect of extended unemployment benefits on non-employment duration is estimated at an average of one to two days, indicating that people receiving unemployment benefits for a longer time period are tending to stay unemployed for a longer spell as well. The results for polynomials of degree 0 to 2 and the non-parametric methods are significant at the 1-percent level, whereas the results for polynomials of order 3 and 4 are only significant at the 5- and 10-percent level, respectively. As already revealed in our simulation study, the standard errors for global polynomial fitting increase with the degree of polynomial used, explaining the larger p-values. Comparing the non-parametric methods, the two bandwidth selection procedures used lead to slightly different estimation results, as the treatment effect estimated with local linear regression and the bandwidth selected by cross-validation leads to a smaller estimate and standard error.

The effect of extended unemployment benefits on the log difference between daily wages in the old and new job is estimated positively as well at an average of 0.004 for global polynomials of degree 1 to 3 and local linear regression using the bandwidth selected by cross-validation. This means, that the re-employment wage is estimated as on average 0.4$\%$ higher for people receiving extended unemployment benefits. The p-values are far larger such that only the polynomial degrees of order 0 to 2 return estimates significant at the 1-percent level. A treatment effect estimation in terms of comparison in means is far off again and even estimates the effect to be negative. Interestingly, when comparing the non-parametric estimation methods, the rule-of-thumb bandwidth selection procedure performs quite differently. It estimates the effect to be nearly zero and is clearly not significant at any conventional level.

As the treatment effect estimates for both outcome variables differ when using the rule-of-thumb bandwidth selection procedure compared to cross-validation in local linear regression, Figure \ref{fig: bw_perf} finally compares the numeric values of the bandwidths selected and plots the treatment effect estimates as well as the 95-percent confidence intervals as a function of the bandwidth. We see that cross-validation always selects the largest possible bandwidth of 10 whereas the rule-of-thumb procedure leads to significantly smaller bandwidths -- 4.64 when estimating the effect on non-employment duration and 1.01 when estimating the effect on re-employment wage. For the outcome non-employment duration, the treatment effect estimated non-parametrically with any of the considered bandwidths always ranges between the one estimated with a linear model and the one estimated with a quadratic polynomial. It slightly increases as a function of the bandwidth until a value of about 6 and then starts marginally decreasing again. When estimating the effect on wage change however, there is a sharp decrease followed by a sharp increase in the treatment effect estimate for small bandwidths and afterwards, the estimate stabilises at the value predicted by the linear and quadratic model as well. In this setting, the rule-of-thumb bandwidth is close to the value minimising the treatment effect estimate.

\begin{figure}[H]
	\centering
	\includegraphics[width=\textwidth]{../../out/figures/data_analysis/treatment_effect_estimates.png}
	\caption{\textsc{Performance of Local Linear Regression with Different Bandwidths.}}
	\label{fig: bw_perf}
	\medskip
	\justify
	\footnotesize{Notes: The bandwidth used in local linear regression is plotted against the corresponding treatment effect estimate and 95-percent confidence intervals. The outcome variable in Panel A is non-employment duration, measured as the number of days between two consecutive jobs. The outcome variable in Panel B is wage change, computed as the log difference between the daily wage in the year of separation and in the year when the new job starts.}
\end{figure}

These findings again highlight that depending on the underlying data generating process, the estimated treatment effect can depend largely on the method used and among non-parametric methods the bandwidth may have a significant impact. Therefore, it is very important to be not too restrictive when choosing the model used for estimation and possibly even report results or various procedures that are in accordance with the visual appearance of the data at hand.


