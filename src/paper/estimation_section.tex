% !TEX root = research_paper.tex

\section{Treatment Effect Estimation} % (fold)
\label{sec: estim}

To estimate the causal effect we want to fit a regression function to the data at hand. Here, we disregard any covariates and regressor values stem from the treatment indicator $D$, running variable $R$ and polynomials thereof. Essentially, we are interested in the difference of the regression function's values when approaching the cutoff from above vis-à-vis below and we allow the regression function to behave differently on the left and right side of the cutoff. Throughout our work, we estimate models of the following kind:
\begin{equation}
Y = \alpha + \tau D + f_{l}(R) + f_{r}(R) + \varepsilon .
\label{eq: model_general}
\end{equation}

Under the identification assumptions \ref{eq: ident_ass}, $\tau$ recovers the causal effect. Further, $\alpha$ is the intercept and $\varepsilon$ constitutes the mean-zero idiosyncratic element. The functions $f_{l}$ and $f_{r}$ are our main objects of interest and we want to investigate how they affect the treatment effect estimator $\tau$.

Identifying causal effects in RDD applications relies by construction on a \textit{locational} feature in the data; treatment assignment discontinuously changes at a specific value of the running variable. This motivates our desire to fit the functional form of the true regression function, especially around the cutoff. In such a setting misspecification of functional forms emerges to be more severe as it may introduce biased estimates of the treatment effect \cite{lee_lemieux}. For intuition suppose the true functional form is nonlinear and we estimate it by means of a linear model. We can still recover a linear prediction which minimizes some criterion, say the sum of squared residuals. While the \textit{best} linear prediction relates to the entire data, at particular points in the regressor's domain we may still be left with serious specification errors. In a RDD study the cutoff may constitute such a specific point, and the estimate would be biased. To reduce specification errors, polynomials of the running variable may be included which can capture (specific) nonlinearities. However, it is not granted that bias due to a misspecified functional form vanishes. Again, note that in RDD the treatment effect inherits a sense of location as it is closely tied to the cutoff determining treatment assignment. Our aim is to get the functional form close to the cutoff right and it is not always clear that data far away from the cutoff can help us to achieve that. In fact, we need to be well informed about the underlying data-generating process to make a case for parametrically estimating the treatment effect using all data available. Hence, the estimation problem in RDD is widely considered to be addressed by nonparametric methods which relax functional form assumptions and pay closer attention to the treatment effect's locational feature by allowing to restrict data to the cutoff's vicinity, \cite{hahn_et_al}) and \cite{lee_lemieux}.

In the following, we formally introduce the parametric and non-parametric models for treatment effect estimation in Regression Discontinuity Design that we are going to compare against each other in our study.


\subsection{Parametric Treatment Effect Estimation} % (fold)
\label{sec: param}
In this project we estimate the treatment effect by means of OLS and allow different functional forms on either side of the RDD cutoff. We implement this by interacting the running variable (and polynomials thereof) with the treatment indicator. As common in RDD applications we center the running variable by subtracting the cutoff \cite{lee_lemieux}. The linear model with polynomial degree one thus writes:
\begin{equation}
Y = \alpha + \tau D + \beta_{l} (R-c) + (\beta_{r} - \beta_{l}) (R-c) D + \varepsilon .
\label{eq: model_param}
\end{equation}

Similarly, the quadratic model is given by
\begin{equation}
Y = \alpha + \tau D + \beta_{l} (R-c) + (\beta_{r} - \beta_{l}) (R-c) D + \gamma_{l} (R-c)^2 + (\gamma_{r} - \gamma_{l}) (R-c)^2 D + \varepsilon .
\end{equation}

In our simulation study and real data application we estimate models with zero up to four polynomial degrees. (Note that a model of zero degree simply returns the difference in means of all individuals below and above the cutoff).



\subsection{Non-Parametric Treatment Effect Estimation} % (fold)
\label{sec: non-param}
To estimate the treatment effect non-parametrically, we rely on local linear regression which fits a linear model to a localised subset of the data and weighs the observations depending on their proximity to the point of interest. The weights are determined by a kernel function $K$ which assigns more weight to observations close to the point of interest. The size of the local neighbourhood used for estimation at a particular point is determined by a non-negative smoothing parameter, the bandwidth $h$.

As stated by \cite{lee_lemieux}, the kernel function has little impact on the treatment effect estimate in practice. But as we are interested in estimating the regression functions at a boundary point -- the cutoff --, we use the boundary optimal triangular kernel $K(r) = \max \lbrace 0, 1 - \vert r \vert \rbrace$ (cf. \cite{cheng_et_al}).

The choice of bandwidth is however more complicated. It restricts the data to be included in the estimation, thus controls the model complexity and involves a trade-off between bias and variance. If the bandwidth is too small we speak of an \textit{undersmoothed} bandwidth which leads to only few observations falling into the local neighbourhood. This results in small bias but large variance and an overfitted density function, highlighting spuriously fine data structures. If the bandwidth is too large, the large amount of observations falling into the local neighborhood leads to small variance but potentially large bias and the density curve might miss important features of the data. In this case we refer to it as an \textit{oversmoothed} bandwidth. Thus, the bandwidth is a crucial ingredient of local linear regression and in the literature on Regression Discontinuity Designs, the following data-driven selection procedures are commonly used: Leave-one-out cross-validation and the rule-of-thumb bandwidth selection procedure by \cite{fan_gij}.

Cross-validation is a procedure to select the optimal smoothing parameter of a regression model (in our case the bandwidth) by partitioning the sample into two subsets, using one to fit the data (the \textit{training set}) and the other one (the \textit{test set}) to test the accuracy of the prediction in terms of mean squared error. Leave-one-out cross-validation is a special case thereof where the test set is of size one, called the \textit{hold-out observation}. To get an estimate for the mean squared error in total, we repeat the procedure of holding-out one observation and predicting its value of the dependent variable again and again until each observation in the sample has served as a hold-out observation once. Conducting these steps for every smoothing parameter in a pre-specified grid, we in the end choose the one that achieves the lowest estimated mean squared error. To be applicable to Regression Discontinuity Designs, \cite{imb_lemieux} and \cite{ludwig_miller} developed a special version where the training set only consists of observations within the bandwidth of the hold-out-observation away from the cutoff. Like this the hold-out observation mimics a boundary point and the selected bandwidth is more appropriate for the boundary estimation setting. Algorithm \ref{alg:cv} gives a detailed description of the cross-validation algorithm used in our study.

\begin{algorithm}[H]
	\caption{Cross-validation bandwidth selection}\label{alg:cv}
	\begin{algorithmic}[1]
		\Require $((R_{i}, Y_{i})_{i \in N}, c, grid) =$ (data on running and dependent variable, cutoff, grid of bandwidths)
		\State $N_{grid} \gets$ number of observations in $grid$
		\State Split the data $(R_{i}, Y_{i})_{i \in N}$ at $c$ into $(R_{i, -}, Y_{i, -})_{i \in N_{-}}$ and $(R_{i, +}, Y_{i, +})_{i \in N_{+}}$
		\State MSE $\gets$ $\left[ 0, \dots, 0 \right]$
		\For{$j = 1, \dots, N_{grid}$}
		\State $h \gets grid[j]$

		\For{$k = 1, \dots, N_{-}$}
		\State $R_{out} \gets R_{k, -}$
		\State $Y_{out} \gets Y_{k, -}$
		\State $\left(R_{train}, Y_{train}\right) \gets$ all observations of $(R_{i, -}, Y_{i, -})_{i \in N_{-}}$ with $R_{out}-h \leq R_{i, -} < R_{out}$
		\State $Y_{predict} \gets$ predicted value of regression function at $R_{out}$ performing a local linear \newline
		\mbox{}\phantom{\textbf{forall} \itshape(} regression with the triangle kernel and bandwidth $h$ of $Y_{train}$ on $R_{train}$
		\State MSE$[j] \gets$ MSE$[j] + \left( Y_{out} - Y_{predict} \right)^{2}$
		\EndFor

		\For{$k = 1, \dots, N_{+}$}
		\State $R_{out} \gets R_{k, +}$
		\State $Y_{out} \gets Y_{k, +}$
		\State $\left(R_{train}, Y_{train}\right) \gets$ all observations of  $(R_{i, +}, Y_{i, +})_{i \in N_{+}}$ with $R_{out} < R_{i, +} \leq R_{out}+h $
		\State $Y_{predict} \gets$ predicted value of regression function at $R_{out}$ performing a local linear \newline
		\mbox{}\phantom{\textbf{forall} \itshape(} regression with the triangle kernel and bandwidth $h$ of $Y_{train}$ on $R_{train}$
		\State MSE$[j] \gets$ MSE$[j] + \left( Y_{out} - Y_{predict} \right)^{2}$
		\EndFor
		\EndFor
		\State $h_{opt} \gets$ value of $grid$ where MSE is minimal
		\State \textbf{return} $h_{opt}$
	\end{algorithmic}
\end{algorithm}

The rule-of-thumb bandwidth selection procedure developed by \cite{fan_gij} for the context of local linear regressions is also based on the concept of mean squared error. The idea is to start with a formula for the optimal bandwidth in terms of an optimal degree of bias and precision and then plug in estimates of unknown quantities. The adaptation to the setting of Regression Discontinuity Designs which we use has been performed by \cite{imbens_kalyanaraman} and is described in Algorithm \ref{alg:rot}.

\begin{algorithm}
	\caption{Rule-of-thumb bandwidth selection}\label{alg:rot}
	\begin{algorithmic}[1]
		\Require $((R_{i}, Y_{i})_{i \in N}, c) =$ (data on running and dependent variable, cutoff)
		\Stepone Estimation of density and conditional variance.
		\State $\sigma^{2}_{R} \gets \frac{1}{N-1} \sum_{i=1}^{N} (R_{i} - \bar{R})^{2}$
		\State $h_{1} \gets 1.84 \cdot \sigma_{R} \cdot N^{-1/5}$
		\State $N_{h_{1}, -} \gets \sum_{i=1}^{N} \mathds{1}_{c-h_{1} \leq R_{i} < c}$
		\State $N_{h_{1}, +} \gets \sum_{i=1}^{N} \mathds{1}_{c \leq R_{i} \leq c+h_{1}}$
		\State $\bar{Y}_{h_{1}, -} \gets \frac{1}{N_{h_{1}, -}} \left(\sum_{i: c-h_{1} \leq R_{i} < c} Y_{i} \right)$
		\State $\bar{Y}_{h_{1}, +} \gets \frac{1}{N_{h_{1}, +}} \left(\sum_{i: c \leq R_{i} \leq c+h_{1}} Y_{i} \right)$
		\State $\widehat{f}_{R}(c) \gets \frac{N_{h_{1}, -} + N_{h_{1}, +}}{2 \cdot N \cdot h_{1}}$ {\color{blue} \Comment{Estimate of the density of $R_{i}$ at $c$}}
		\State $\widehat{\sigma}^{2}(c) \gets \frac{1}{N_{h_{1}, -} + N_{h_{1}, +}} \left( \sum_{c-h_{1} \leq R_{i} < c} \left( Y_{i} - \bar{Y}_{h_{1}, -}\right)^{2} + \sum_{i: c \leq R_{i} \leq c+h_{1}} \left( Y_{i} - \bar{Y}_{h_{1}, +} \right)^{2} \right)$
		\newline {\color{blue} \Comment{Estimate of the conditional variance of $Y_{i}$ given $R_{i}=r$ at $r=c$}}

		\Steptwo Estimation of second derivatives.
		\State $N_{-} \gets$ number of observations with $R_{i} < c$
		\State $N_{+} \gets$ number of observations with $R_{i} \geq c$
		\State $median(R_{-}) \gets$ median of observations with $R_{i} < c$
		\State $median(R_{+}) \gets$ median of observations with $R_{i} \geq c$
		\State Temporarily discard observations with $R_{i} < median(R_{-})$ or $R_{i} > median(R_{+})$ and estimate the regression function \newline $Y_{i} = \alpha_{0} + \alpha_{1} \cdot \mathds{1}_{R_{i} \geq c} + \alpha_{2} \cdot (R_{i}-c) + \alpha_{3} \cdot (R_{i}-c)^{2} + \alpha_{4} \cdot (R_{i}-c)^{3} + \varepsilon_{i}$
		\State $\widehat{m}_{3}(c) \gets 6 \cdot \widehat{\alpha}_{4}$
		\State $h_{2, -} \gets 3.56 \left( \frac{\widehat{\sigma}^{2}(c)}{\widehat{f}(c) \cdot \max\lbrace \left(\widehat{m}_{3}(c)\right)^{2}, 0.01\rbrace}\right)^{1/7} N_{-}^{-1/7}$
		\State $h_{2, +} \gets 3.56 \left( \frac{\widehat{\sigma}^{2}(c)}{\widehat{f}(c) \cdot \max\lbrace \left(\widehat{m}_{3}(c)\right)^{2}, 0.01\rbrace}\right)^{1/7} N_{+}^{-1/7}$

		\State $(R_{i, h_{2, -}}, Y_{i, h_{2, -}}) \gets$ observations with $c-h_{2, -} \leq R_{i} < c$
		\State $(R_{i, h_{2, +}}, Y_{i, h_{2, +}}) \gets$ observations with $c \leq R_{i} \leq c+h_{2, +}$
		\State $N_{h_{2}, -} \gets$ number of observations with $c-h_{2, -} \leq R_{i} < c$
		\State $N_{h_{2}, +} \gets$ number of observations with $c \leq R_{i} \leq c+h_{2, +}$
		\State Estimate the regression function  $Y_{i, h_{2,-}} = \beta_{0} + \beta_{1} (R_{i, h_{2, -}}-c) + \beta_{2} (R_{i, h_{2, -}}-c)^{2} + \epsilon_{i}$
		\State $\widehat{m}^{(2)}_{-}(c) \gets 2 \cdot \widehat{\beta}_{2}$ {\color{blue} \Comment{Estimate of the curvature of the regression function left of $c$}}
		\State Estimate the regression function  $Y_{i, h_{2,+}} = \gamma_{0} + \gamma_{1} (R_{i, h_{2, +}}-c) + \gamma_{2} (R_{i, h_{2, +}}-c)^{2} + \epsilon_{i}$
		\State $\widehat{m}^{(2)}_{+}(c) \gets 2 \cdot \widehat{\gamma}_{2}$ {\color{blue} \Comment{Estimate of the curvature of the regression function right of $c$}}

		\Stepthree Calculation of regularisation terms and optimal bandwidth.
		\State $\widehat{r}_{-} \gets$ $\frac{720 \cdot \widehat{\sigma}^{2}(c)}{N_{h_{2}, -} \cdot h_{2, -}^{4}}$
		\State $\widehat{r}_{+} \gets$ $\frac{720 \cdot \widehat{\sigma}^{2}(c)}{N_{h_{2}, +} \cdot h_{2, +}^{4}}$
		\State $h_{opt} \gets 3.4375 \cdot \left(\frac{2 \cdot \widehat{\sigma}^{2}}{\widehat{f}(c) \cdot \left( \left( \widehat{m}^{(2)}_{+}(c) - \widehat{m}^{(2)}_{-}(c) \right)^{2} + \left( \widehat{r}_{+} + \widehat{r}_{-} \right) \right)}\right) \cdot N^{-1/5}$
		\State \textbf{return} $h_{opt}$
	\end{algorithmic}
\end{algorithm}

Using these procedures to select the bandwidth $h$, we follow the approach by \cite{fan_gij} and transform the local linear regression model into a standard weighted least squares problem where weights are determined by the kernel. According to \cite{lee_lemieux} we can then use the following pooled regression model and a weighted least squares regression to estimate the treatment effect.
\begin{align}
Y = \alpha + & \tau D + \beta_{l} (R-c) + (\beta_{r} - \beta_{l}) (R-c) D + \varepsilon \\
&\text{ where } c - h \leq R \leq c + h \nonumber
\label{eq: model_non_param}
\end{align}


Parametric and non-parametric treatment effect estimation, as presented here, are fundamentally different since non-parametric methods relax functional form assumptions, constrain the dataset and face different statistical properties. However, in a sense, the non-parametric model can be understood as a generalization of the parametric model since for given data where each observation is weighted equally and when the bandwidth coincides with the regressor's support the non-parametric and parametric model are in fact the same. Hence, parametric and non-parametric approaches are sometimes called \textit{global} and \textit{local} methods, respectively. That being said, we discuss the trade-offs between these two methods and highlight them in our simulation study and real data application in Sections \ref{sec:sim_study} and \ref{sec: data_application}, respectively.
